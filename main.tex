\documentclass{jsarticle}

\usepackage{listings}

\title{ {\LaTeX} 自動ビルド・校正レビュー機能付きテンプレート}
\author{pddg}
\date{\today}

\newcommand{\code}[1]{\texttt{#1}}

\begin{document}
\maketitle

\section{概要}

\subsection{このテンプレートの目的}

これは日本語論文を書く上で

\begin{itemize}
    \item LaTeXの環境構築
    \item 誤字・表記揺れの修正
\end{itemize}

のような本質的でない部分に時間をかけず,論文の内容に集中するためのものです.

\subsection{対象となるユーザ}
理工系の卒業論文・修士論文を書く人を対象としています(一部設定をカスタマイズすることでそれ以外の分野でもお使い頂けます).
また,GitおよびGitHubを使用してバージョン管理を行うことで真価を発揮するため,Gitの使用は前提となります.


これはどんなTeXソースでもビルド出来るようにすることは目標としていないため,

\begin{itemize}
    \item デフォルトのままで完璧に上手くやって欲しい
    \item 自分でカスタマイズはしたくない
\end{itemize}

というユーザにはあまり向かない可能性があります.

\subsection{提供する機能一覧}

ホストOSへインストールされているLaTeXおよび,TeX LiveをインストールしたDockerコンテナを使用し,下記の操作をサポートしています.

\begin{itemize}
    \item latexmkを使用したPDFのビルド(\code{make pdf}または\code{make docker})
    \item TeXソースの変更検知時に自動ビルド(\code{make watch}または\code{make docker-watch}
    \item 生成物の破棄(\code{make clean})
\end{itemize}

また,GitHub Actionsを用いて下記の操作をサポートします.

\begin{itemize}
    \item masterブランチへのpush時に自動ビルド・生成PDFをアップロード
    \item Pull Request作成時にルールベースでの校正およびレビューでの指摘
    \item タグをpushするたびにGitHub Releasesへ自動デプロイ
\end{itemize}

\section{使用方法}

\subsection{動作する環境}

検証したOSは以下の通りです.

\begin{itemize}
    \item Windows 10
    \item macOS 10.14 or later
    \item Ubuntu 18.04 LTS or later
\end{itemize}

Windows・UbuntuではTeX Live 2019を,macOSではMacTeX 2019を用いています.その他のツールの要求バージョンは以下の通りです.

\begin{description}
    \item[Docker] \\
        Docker CE 18.09 or later
    \item[Make] \\
        GNU Make 3.8 or later
    \item[hub(https://github.com/github/hub)] \\
        hub version 2.12.0 or later
\end{description}

ただし,これは完璧に動作することを保証したものではありません.使用者の環境によっては一部機能等がうまく動作しない可能性もあります.

\end{document}
